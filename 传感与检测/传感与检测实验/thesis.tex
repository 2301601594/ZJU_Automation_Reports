\PassOptionsToPackage{quiet}{fontspec}
\documentclass[12pt,a4paper,UTF8]{article}
\usepackage{thesis} % 格式控制
\usepackage{indentfirst}
\usepackage{geometry}  % 页面布局
\usepackage{multirow}  % 合并行
\usepackage{makecell}  % 单元格换行
\usepackage{xcolor}    % 颜色支持
\usepackage{array}     % 表格列格式



\setlength{\parindent}{2em} % 控制首行缩进  
\addtolength{\parskip}{3pt} % 控制段落距离  
\onehalfspacing % 1.5倍行距  
\graphicspath{{./figures/}} % 指定图片所在文件夹  


\classname{传感与检测}  % 设置课程名称
\makepagestyle{}{\printclassname ~实验报告}

\begin{document}
\maketitlepage{传感与检测实验}{教10 6005、6007}{董佳泽}{3230104322}{\today}{赵久强、曹峥、冯毅萍} %封面页 

\maketoc    %目录页

\section{温度检测与显示仪表实验}
温度是国际单位制(SI)中的七个基本物理量之一,在各个相应学科单位制
中占有重要的地位,因此温度检测在工业生产、科学研究和日常生活中等方面都
具有广泛的应用。热电阻、热电偶是目前温度检测过程中应用比较广泛的两种测
温传感器,其具有测量范围大、性能稳定、测量精度高、复现性好、互换性好、
结构简单、安装使用方便、信号便于远传、自动记录和集中监控等主要特点。因
此,本次温度检测实验重点围绕热电阻和热电偶两类温度传感器开展实施,主要
包含温度检测传感器与检测仪表实验、热电偶检定实验和热电偶温度显示仪表实
验三部分
\subsection{温度检测传感器与检测仪表实验}
\label{1-1}
\subsubsection{实验目的}
\begin{enumerate}
  \item 使学生掌握工业现场常用的温度传感器热电偶和热电阻测温的基本原理及使用方法;
  \item 使学生掌握热电偶冷端温度补偿的基本概念及处理方法;
  \item 使学生掌握热电偶、热电阻测温系统的结构形式和接线方式;
  \item 使学生掌握热电偶、热电阻检测仪表的基本使用方法。
\end{enumerate}
\subsubsection{实验要求}
\begin{enumerate}
  \item 掌握热电偶、热电阻检测温度的原理及其使用方法;
  \item 掌握热电偶测温过程中的冷端温度补偿的基本概念和处理方法;
  \item 掌握热电偶、热电阻测温系统的结构形式和接线方式;
  \item 了解实验室仪器设备的作用和使用方法,完成接线并使用检测仪表进行测量;
  \item 实验报告应包括测量原理图、接线图、测量误差分析等。
\end{enumerate}
\subsubsection{实验原理}

电阻式检测元件在检测技术领域具有广泛应用,其基本原理是将被测物理量
转化为电阻值的变化,然后利用测量电路测出电阻的变化值,从而达到对被测物
理量检测的目标。目前使用的金属热电阻材料有铂、铜、镍、铁等,其中最为广
泛的为铂、铜材料,并已经实现了标准化生产,且具有较高的稳定性和准确度。
以铂材料电阻为例,铂电阻的电阻值与温度的关系是一个典型的非线性函数,一
般工业用的铂电阻可以用下式表示:
$$
R_t = R_0 (1 + At + Bt^2 + C(t - 100)t^3) \quad (0^\circ\mathrm{C} < t < 850^\circ\mathrm{C})
$$
$$
R_t = R_0 \left\{1 + At + Bt^2 + C[t(t - 100)]\right\} \quad (-200^\circ\mathrm{C} < t < 0^\circ\mathrm{C})
$$
式中, $R_t$为温度在$t$ °C 时铂电阻的电阻值;$A$、$B$、$C$为常数。


热电式检测元件是利用敏感元件将温度变化转化为电量的变化,从而达到测
量温度的目的。最典型的热电式检测元件为热电偶,目前也常用于工业现场的测
温环节,具有结构简单、测温准确度高和测温范围广等优势。如图\ref{redianou}热电偶结
构示意图所示,将两种不同的导体$A$、$B$ 连接成闭合回路,将它们的两个接点分
别置于温度为$T$ 及$T_0$的热源中,则在该回路内就会产生热电动势。
    \begin{figure}[!htbp]
        \centering
        \includegraphics[width=0.8\textwidth]{热电偶结构示意图.png}
        \caption{热电偶结构示意图}
        \label{redianou}
    \end{figure}
\subsubsection{实验步骤}
\begin{enumerate}
  \item 通电前,首先认识实验装置中各个组成部分仪表的功能。按下表进行接线并检查。

\begin{table}[htbp]
\begin{center}
\caption{温度检测接线表}
\begin{tabular}{|c|c|c|}
\hline
始——终               & 导线规格    & 备注        \\ \hline
标准热电偶-ET3805 输入端5  & 配套补偿导线  & 规格二等标准热电偶 \\ \hline
E 型热电偶-ET3805 输入端1 & 配套补偿导线  & 接①中间测量插孔  \\ \hline
K 型热电偶-ET3805 输入端2 & 配套补偿导线  & 接②中间测量插孔  \\ \hline
普通热电阻-ET3805 输入端3  & 配套热电阻引线 & 四线制接线方式   \\ \hline
AC 电源-ET3805 电源插头  & 三芯橡胶护套线 & 电源线,必须接地  \\ \hline
\end{tabular}
\end{center}
\end{table}
  \item 检查完毕后,打开ET3805 干体炉电源开关,此时电源指示灯常亮,等待设备启动完毕。
  \item 在主界面设置温度100℃,设置好温度后,在主界面启动仪器。
  \item 等待“测量温度”文字呈绿色显示,默认的温度稳定条件为波动度0.4℃,目标偏差0.5℃,稳定时间5min(满足波动度和目标偏差要求后的持续时间)。
  \item “测量温度”文字呈绿色后,分别记录标准、普通热电偶以及热电阻的输出值(按照实验要求记录电信号数值)
  \item 将E、K 型热电偶从干体炉输入端拔出,将E 型热电偶插入干体炉输入端,待稳定片刻后观察并记录此时的显示数据,完成思考题3 并用记录数据验证解题思路是否正确。
  \item 根据步骤6 所记录的数据,查阅分度表,判断标准热电偶的分度号,参考标准热电偶的输出值,分别计算该温度点下普通E、K 型热电偶以及热电阻(pt100)的温度相对误差。


\end{enumerate}
\subsubsection{数据记录与处理}
测量时读取自动冷端温度(室温)为21.74℃,将测量结果记录在表2中。
\begin{table}[htbp]
    \centering
    \caption{温度传感器实验数据记录表}
    % 定义表格列格式:加 | 表示竖线,c 表示居中
    \begin{tabular}{|c|c|c|c|c|c|}
        \hline
        \multirow{2}{*}{序号} & 设定温度 & 标准热电偶 & E 型热电偶 & K 型热电偶 & 热电阻电阻值/ \\
        & 值/$^\circ$C & 电压值/mV & 电压值/mV & 压值/mV & $\Omega$ \\
        \hline
        1 & \multirow{4}{*}{100.000} & 0.5116 & 4.8601 & 3.1102 & 137.8190 \\ \cline{1-1} \cline{3-6}
        2 &                          & 0.5118 & 4.8602 & 3.1100 & 137.8200 \\ \cline{1-1} \cline{3-6}
        3 &                          & 0.5119 & 4.8600 & 3.1095 & 137.8270 \\ \cline{1-1} \cline{3-6}
        4 &                          & 0.5117 & 4.8595 & 3.1098 & 137.8320 \\
        \hline
    \end{tabular}
\end{table}

根据表2内容对数据进行处理,将结果记录在表3中。

\begin{table}[htbp]
    \centering
    % 设置行高,使表格看起来不那么拥挤
    \renewcommand{\arraystretch}{1.3}
    \caption{温度传感器实验数据处理表}
    \begin{tabular}{|c|c|c|c|c|}
        \hline
        处理方式 & {标准热电偶} & {E型热电偶} & {K型热电偶} & {热电阻} \\
        \hline
        求取均值 & 0.5188 & 4.86 & 3.1099 & 137.8245 \\
        \hline
        冷端查表 & 0.119 & 1.252 & 0.838 & 108.18 \\
        \hline
        数值换算 & 0.6308 & 6.112 & 3.9479 & 137.8245 \\
        \hline
        温度查表 & 98 & 97 & 96 & 98 \\
        \hline
        相对误差 & 0 & 1\% & 2\% & 0 \\
        \hline
    \end{tabular}
\end{table}

由于书后附录的表只能精确到温度的个位,因此在查表对应温度时存在一定误差。

\subsubsection{思考题}
\begin{problem}
热电阻式检测元件主要分为几种,各适用于哪种场合?
\end{problem}
\begin{solution}
主要分为\textbf{金属热电阻}和\textbf{半导体热敏电阻}两大类,其中金属热电阻主要为\textbf{比铂电阻}和\textbf{铜电阻}。
\begin{itemize}
  \item 铂电阻:物理化学性能极稳定,精度高,复现性好。
  \begin{itemize}
    \item 适用场景:工业级的高精度温度测量,作为标准温度计使用。适用于 $-200^\circ\text{C} \sim 850^\circ\text{C}$ 的宽温区。
  \end{itemize}
  
  \item 铜热电阻:价格便宜,线性度极好,但易氧化。
  \begin{itemize}
    \item 适用场合: 测量精度要求不高,且温度较低($-50^\circ\text{C} \sim 150^\circ\text{C}$)的非腐蚀性介质环境(如电机绕组测温)。
  \end{itemize}

  \item 半导体热敏电阻:灵敏度比金属高 10~100 倍,体积小,响应快,但线性度差,互换性差。
    \begin{itemize}
    \item 适用场合: 家电温控(空调、冰箱)、电路过热保护、快速响应的点温测量、PCB板级测温。
  \end{itemize}
\end{itemize}
\end{solution}

\begin{problem}
试比较热电偶测温和热电阻测温有哪些区别(可以从原理、系统构成和应用
场合分析)。
\end{problem}
\begin{solution}
\begin{enumerate}
  \item \textbf{工作原理}:热电偶利用压热效应,属于有源传感器,不需要外接激励。热电阻是利用电阻的热效应,是无源传感器,需要外接电源。
  \item \textbf{系统构成}:热电偶需要进行冷端补偿,且使用特定的补偿导线延展信号。热电阻可以使用普通的铜导线,但是需要考虑引线电阻的影响。
  \item \textbf{应用场合}:热电偶用于测量高温,热电阻一般用于测量中低温。
\end{enumerate}
\end{solution}

\begin{problem}
有一配K 分度号的电子电位差计,在测温过程中配错了E 分度号的热电偶,
此时仪表指示为315℃,假设冷端温度为23℃,则实测温度约为多少℃(取整数)。
\end{problem}
\begin{solution}
  仪表显示的温度是基于K型分度表计算的,需先还原为实际输入毫伏值,再结合E型热电偶特性反推真实温度。

  \textbf{步骤 1:计算仪表输入端的实际热电势 $V_{in}$}
  \begin{align*}
    V_{in} &= E_K(T_{\text{示}}) - E_K(T_0) \\
    &= E_K(315^\circ\text{C}) - E_K(23^\circ\text{C}) \\
    &\approx 12.83\,\text{mV} - 0.92\,\text{mV} \\
    &= 11.91\,\text{mV}
  \end{align*}

\textbf{步骤 2:计算E型热电偶产生的总热电势 $E_E(T_{\text{真}})$}
\begin{align*}
    E_E(T_{\text{真}}) &= V_{in} + E_E(T_0) \\
    &= 11.91\,\text{mV} + E_E(23^\circ\text{C}) \\
    &\approx 11.91\,\text{mV} + 1.38\,\text{mV} \\
    &= 13.29\,\text{mV}
\end{align*}

\textbf{步骤 3:查表反推真实温度}
\noindent 查 E 型热电偶分度表可知:
$$ E_E(190^\circ\text{C}) \approx 12.70\,\text{mV}, \quad E_E(200^\circ\text{C}) \approx 13.42\,\text{mV} $$
由于 $13.29\,\text{mV}$ 介于两者之间,利用线性插值法计算:
\begin{align*}
    T_{\text{真}} &\approx 190 + \frac{13.29 - 12.70}{13.42 - 12.70} \times 10 \\
    &\approx 198.2^\circ\text{C}
\end{align*}

\noindent \textbf{结论:} 实测温度约为 $198^\circ\text{C}$。

  在实验中,我们将E和K热电偶进行互换读数,读到$E_K=72.001$℃、$E_E=140.484$℃,通过以上的方法可以求出,两者对应的温度分别为$T_E=107$℃、$T_K=91$℃
\end{solution}

\begin{problem}
简述热电阻测温有哪几种接线方式,以及分析各自优缺点。
\end{problem}
\begin{solution}
热电阻测温主要有三种接线方式,分别为二线制、三线制和四线制。

\begin{enumerate}
    \item \textbf{二线制}
    \begin{itemize}
        \item \textbf{方式:} 热电阻两端各连接一根导线。
        \item \textbf{优点:} 结构最简单,成本最低。
        \item \textbf{缺点:} 导线电阻直接计入测量结果,导致测量值偏高,误差较大。
        \item \textbf{适用:} 精度要求不高、且引线较短的场合。
    \end{itemize}

    \item \textbf{三线制}
    \begin{itemize}
        \item \textbf{方式:} 热电阻一端接两根线,另一端接一根线。
        \item \textbf{优点:} 配合电桥电路,在三根导线阻值相等的前提下,可有效抵消导线电阻的影响。
        \item \textbf{缺点:} 需要保证三根导线的材质、长度和线径一致。
        \item \textbf{适用:} 绝大多数工业现场的温度测量。
    \end{itemize}

    \item \textbf{四线制}
    \begin{itemize}
        \item \textbf{方式:} 热电阻两端各接两根线(两根电流线,两根电压线)。
        \item \textbf{优点:} 利用恒流源和高阻抗电压测量,可完全消除导线电阻的影响,精度最高。
        \item \textbf{缺点:} 导线数量多,成本较高,电路复杂。
        \item \textbf{适用:} 实验室高精度测量或作为标准温度计。
    \end{itemize}
\end{enumerate}
\end{solution}
\subsection{热电偶校准实验}
\subsubsection{实验目的}
\begin{enumerate}
  \item 使学生熟悉热电偶校准的设备、规程及热电偶校准的方法;
  \item 使学生掌握热电偶冷端温度补偿、补偿导线等的基本概念及应用方法;
  \item 使学生了解温度控制系统的基本构成及控制精度的概念;
  \item 使学生了解测量系统的动态特性的实验研究方法。
\end{enumerate}
\subsubsection{实验要求}
\begin{enumerate}
  \item 了解热电偶的结构,掌握热电偶检测温度的原理及其使用方法;
  \item 掌握热电偶传感器校准的方法和操作过程,了解所用仪器的选型和用法等;
  \item 掌握热电偶测温中的几个关键概念,如热电偶冷端处理、补偿导线的原理;
  \item 实验报告应包括测量原理图、接线图、测量误差分析等。
\end{enumerate}
\subsubsection{实验原理}
热电偶使用一段时间后,测量端由于氧化腐蚀和高温下的再结晶等因素,热
电特性会发生一定变化,因而会产生测量误差,为了确保其测量准确度,必须对其进行校准或者检定。一般采用\textbf{比较法}进行校准,本实验将标准热电偶和被校热电偶的测量端同时插入干体炉的均热块中(尽量使其测量端的温度相同),待干体炉的控温系统将均热块的温度控制在变化不超过正负0.2℃时直接测量标准热电偶与被校热电偶的热电势,通过比较、换算,最后确定被校热电偶的示值误差,并判断是否超差。
\subsubsection{实验步骤}
\begin{enumerate}
  \item 首先按照实验\ref{1-1}中步骤进行正确接线
  \item 点按菜单界面的“控温参数设置”—>点击“控温历史数据”—>点击“数据列表”—>点击“删除”后依次将历史数据全部删除。
  \item 返回菜单界面,点按菜单界面的“任务”按键,进入任务界面。点击“任务记录” —>点击“删除” 后依次将检定数据全部删除。
  \item 返回菜单界面,再次点按菜单界面的“任务”按键,进入任务界面。任务功能主要用于被检设备的智能校准或检定。等待完成后用U 盘导出需要的实验数据。
  \item \textbf{由于时间关系,本次实验校准的温度点设置为(200℃,250℃,300℃)。}
\end{enumerate}

\subsubsection{数据记录与处理}
将仪器校准得到的数据下载到U盘,通过软件处理,得到结果如表4。
\begin{table}[htbp]
    \centering
    \caption{热电偶校准实验数据记录表}
    \small % 使用小号字体以适应宽度
    \renewcommand{\arraystretch}{1.4} % 增加行高,防止拥挤
    
    \begin{tabular}{|c|c|c|c|c|c|}
        \hline
        校准地点 & \multicolumn{3}{c|}{教10 6005} & 校准时间 & \\
        \hline
        参考标准 & 外置标准温度计 & 型号规格 & S & 序列号 & 1 \\
        \hline
        \multicolumn{6}{|c|}{\textbf{被校热电偶信息}} \\
        \hline
         & & No.1 & No.2 & No.3 & No.4 \\
        \hline
        型号 & & 1 & 2 & 3 & \\
        \hline
        分度号 & & E & K & PT100-385 & \\
        \hline
        量程/$^\circ$C & & -40$\sim$1000 & -40$\sim$1000 & -40$\sim$1000 & \\
        \hline
        制造单位 & & & & & \\
        \hline
        出厂编号 & & & & & \\
        \hline\hline
        % 数据部分表头
        \multirow{2}{*}{\makecell{校准温度点\\/$^\circ$C}} & 
        \multirow{2}{*}{\makecell{标准温度\\(温度$^\circ$C/电信号mV)}} & 
        \multicolumn{4}{c|}{\makecell{被检热电偶温度 \quad (温度$^\circ$C/电信号mV)}} \\
        \cline{3-6}
         & & \multicolumn{1}{c|}{No.1} & \multicolumn{1}{c|}{No.2} & \multicolumn{1}{c|}{No.3} & \multicolumn{1}{c|}{No.4} \\ 
         % 注:原图此处没有再次列出No.1-4,但为了对齐数据,逻辑上是对应的
        \hline
        \textcolor{red}{200} & 195.41/1.2500 & 189.79/11.2856 & 192.18/6.9009 & 189.61/172.0278 & \\
        \hline
        冷端温度 & 23.15 & 23.15 & -- & -- & \\
        \hline
        误差 & -- & 5.63 & 3.23 & 5.8 & \\
        \hline\hline
        
        \textcolor{red}{250} & 248.45/1.7008 & 243.78/15.3164 & 246.11/9.0643 & 243.28/191.6646 & \\
        \hline
        冷端温度 & 23.28 & 23.28 & 23.28 & -- & \\
        \hline
        误差 & -- & 4.67 & 2.35 & 5.17 & \\
        \hline\hline
        
        \textcolor{red}{300} & 298.75/2.1444 & 294.20/19.1723 & 296.23/11.1072 & 293.39/209.6943 & \\
        \hline
        冷端温度 & 23.64 & 23.64 & 23.64 & -- & \\
        \hline
        误差 & -- & 4.55 & 2.52 & 5.36 & \\
        \hline
    \end{tabular}
\end{table}

对校准过程中的4种温度传感器的测得的温度数据进行处理,绘制成以下曲线图。

\begin{figure}[!htbp]
    \centering
    \subcaptionbox{200℃}[0.33\textwidth][c]{
        \centering
        \includegraphics[width=0.32\textwidth]{200.png}
         
    }%
        \subcaptionbox{250℃}[0.33\textwidth][c]{
        \centering
        \includegraphics[width=0.32\textwidth]{250.png}
         
    }%
        \subcaptionbox{300℃}[0.33\textwidth][c]{
        \centering
        \includegraphics[width=0.32\textwidth]{300.png}
         
    }%
    \caption{校准温度曲线}
     
\end{figure}
\clearpage
\subsubsection{数据分析}
\begin{enumerate}
  \item \textbf{校准结论}
  
  在表4中可以看出,所有被测传感器的测量值都低于标准温度值,即存在负偏差。
  
  分别对3种传感器进行分析,可知:
  K型热电偶的表现相对最好,E型热电偶存在较大的误差,热电阻传感器误差最大。因此使用这些传感器对温度进行读数时,需要加入一个正向的修正值。
  \item \textbf{误差分析}
  \begin{itemize}
    \item 如果仪表测量的冷端温度不准确,那会带来整体的偏差。
    \item 实际使用的电偶丝与国家标准不可能完全一致,这会带来误差。
    \item 热电偶插入的深度可能不够,导致外部热量沿保护管散失。
  \end{itemize}
  \item \textbf{曲线变化}
  从曲线中也可看出,三种传感器都存在一定程度的负偏差,测量得到的温度都小于标准热电偶。
\end{enumerate}
\subsubsection{思考题}
\begin{problem}
列举冷端补偿的方法。
\end{problem}
\begin{solution}
\begin{itemize}
    \item \textbf{冰浴法:} 将冷端置于 $0^\circ\text{C}$ 的冰水混合物中。
    \item \textbf{计算修正法:} 利用公式 $E(T, 0) = E(T, T_0) + E(T_0, 0)$ 进行数值修正。
    \item \textbf{电桥补偿法:} 利用不平衡电桥产生的电压来抵消冷端温度变化带来的电势变化。
    \item \textbf{仪表自动补偿法:} 现代仪器通过内置热敏电阻测量冷端温度并自动软件补偿。
\end{itemize}
\end{solution}

\begin{problem}
在热电偶测温电路中采用补偿导线时,应如何连接,需要注意哪些问题?
\end{problem}
\begin{solution}
\begin{itemize}
    \item \textbf{连接方式:} 正极接正极,负极接负极(注意:部分标准中红色绝缘层可能为负极,需仔细判别)。
    \item \textbf{注意事项:}
    \begin{enumerate}
        \item \textbf{型号匹配:} 补偿导线的分度号必须与热电偶一致(如 K 型偶配 K 型线)。
        \item \textbf{极性正确:} 极性接反会造成双倍的测量误差。
        \item \textbf{温度一致:} 补偿导线与热电偶连接点的两个接线端温度必须保持一致。
        \item \textbf{抗干扰:} 信号线应尽量短,并避开强磁场或动力线,尽量使用屏蔽线。
    \end{enumerate}
\end{itemize} 
\end{solution}

\begin{problem}
当补偿导线类型和极性混淆不明时如何判别?
\end{problem}
\begin{solution}
\begin{itemize}
    \item \textbf{判别极性(热水法):} 
    将导线一端的两根线绞合,放入热水中;另一端接入万用表毫伏档。若电压为正,则与红表笔相连的导线为正极。
    
    \item \textbf{判别类型(比对法):}
    在确定极性后,测量其在特定温差下(如 $100^\circ\text{C}$ 沸水与室温)产生的热电势大小,与标准分度表比对。例如,K型线产生的电势通常小于 E型线。
    
    \item \textbf{外观识别:} 
    (辅助方法)查看绝缘层颜色,依据国标(GB/T 4989)或美标(ASTM)色谱进行初步判断。
\end{itemize} 
\end{solution}

\begin{problem}
试画出典型计算机控制系统的单回路控制方块图。
\end{problem}
\begin{solution}
    \begin{figure}[!htbp]
        \centering
        \includegraphics[width=0.5\textwidth]{闭环反馈.png}
        \caption{单回路控制方块图}
    \end{figure}
\end{solution}
\subsection{热电偶温度显示仪表实验}

\subsubsection{实验目的}
\begin{enumerate}
  \item 掌握热电偶温度显示记录仪表的原理、构造及使用方法;
  \item 掌握典型温度显示记录仪表精度校准的基本方法;
  \item 了解目前主流过程校验仪的功能,并掌握其基本使用方法;
  \item 掌握显示仪表的误差分析及准确度等级判断。
\end{enumerate}
\subsubsection{实验器材}
\begin{enumerate}
  \item 被校热电偶温度显示记录仪表一台(\textbf{量程0~1300℃})
  \item 便携式过程校验仪一台
  \item 高精度温度显示表(室温参考)
  \item 实验原理参考实验总体介绍
\end{enumerate}
\subsubsection{实验步骤}
\begin{enumerate}
  \item 按照图\ref{xianshijiluyibiao}红框所示接好显示仪表输入端导线,将温度显示仪表接通电源。
    \begin{figure}[!htbp]
        \centering
        \includegraphics[width=0.5\textwidth]{显示记录仪表输入端.jpeg}
        \caption{显示记录仪表接线图}
        \label{xianshijiluyibiao}
    \end{figure}
    \item 显示仪表在工作状态下,同时长按方向键“上键”与“下键”,在密码确认后即进入输入设置(无密码,自动保存上一次设置)。在“输入设置”下按“ok”键进入通道设置。
    \item 在“输入设定”下,调整光标到信号类型,当光标对准“型号类型”时,按“ok”键,本实验选定信号类型为:K,按照提示保存确认即可(如已经设置好,则仅需要确认即可)。
    \item 便携式过程校验仪在开机状态下,按MEA/SOUR 键切换至输出模式,按键调整到mV 电压输出工作状态(长按对应数字键进行切换)。
    \item 按热电偶测量功能接线示意图所示把校验仪的输出端子接到显示记录仪表上。
    \item 检查热电偶温度显示仪表的量程范围(量程0~1300℃),在全量程范围中均匀选取10 个温度基准点进行精度校验。
    \item 选好温度基准点后,查阅热电偶分度表换算为相应的mV 电信号(注意显示记录仪表自带冷端温度补偿的影响)。
    \item 依次在过程校验仪上的数字键输入需要输出的电信号数值,按确认键后,待显示稳定后记录实验数据。

\end{enumerate}
\clearpage
\subsubsection{数据记录与处理}
\begin{table}[htbp]
    \centering
    \caption{实验数据记录表}
    % \caption{温度校验数据表} % 如需标题可取消注释
    
    % 设置行高为 1.5 倍,使表格看起来像您的模板一样宽敞
    \renewcommand{\arraystretch}{1.5}
    
    % 定义列格式:|c| 表示每列之间都有竖线,内容居中
    \begin{tabular}{|c|c|c|c|c|}
        \hline
        序号 & 校验温度基准点 & 过程校验仪输出电压值 & 温度记录显示仪表读数 & 误 差 \\
        \hline
        1 & 0$^\circ$C & -0.798 mV & -0.786 mV & 1.5038\% \\
        \hline
        2 & 144$^\circ$C & 5.098 mV & 5.106 mV & 0.1569\% \\
        \hline
        3 & 289$^\circ$C & 10.955 mV & 10.960 mV & 0.0456\% \\
        \hline
        4 & 433$^\circ$C & 16.954 mV & 16.960 mV & 0.0354\% \\
        \hline
        5 & 578$^\circ$C & 23.172 mV & 23.175 mV & 0.0129\% \\
        \hline
        6 & 722$^\circ$C & 29.209 mV & 29.211 mV & 0.0068\% \\
        \hline
        7 & 867$^\circ$C & 35.202 mV & 35.204 mV & 0.0057\% \\
        \hline
        8 & 1011$^\circ$C & 40.906 mV & 40.905 mV & 0.0024\% \\
        \hline
        9 & 1156$^\circ$C & 46.42 mV & 46.421 mV & 0.0022\% \\
        \hline
        10 & 1300$^\circ$C & 51.612 mV & 51.610 mV & 0.0039\% \\
        \hline
    \end{tabular}
\end{table}
  
计算可得,仪表的最大绝对误差为$\Delta_{max}=0.12$mV,量程为$52.41$mV

因此代入引用误差的计算公式有:

$$\gamma = \frac{0.012}{52.41} \times 100\% \approx \mathbf{0.0229\%}$$

根据数据可以该仪表的精确度等级为0.05级

\subsubsection{思考题}
\begin{problem}
当温度显示仪表热电偶输入端的输入电势为0mV 时,指示温度应该是多少;
当温度显示仪表没有输入时,指示值又是多少;将温度显示仪表热电偶输入
端短接,指示值又是多少?
\end{problem}
\begin{solution}
\begin{itemize}
        \item \textbf{当输入电势为 0mV 时:} 
        仪表指示温度应为\textbf{室温(环境温度)}。
        
        \item \textbf{当仪表没有输入(开路)时:}
        仪表通常指示\textbf{最大量程、溢出符号(如 OVER/Err)或无穷大}。
        
        \item \textbf{当输入端短接时:}
        仪表指示温度应为\textbf{室温(环境温度)}。
    \end{itemize}
\end{solution}

\begin{problem}
如何判断热电偶温度显示仪表是否自带冷端温度补偿?
\end{problem}
\begin{solution}
将仪表的热电偶输入端\textbf{短路}。
    \begin{itemize}
        \item 如果仪表指示值约为\textbf{室温}(例如 $25^\circ\text{C}$),则说明仪表\textbf{有}冷端补偿功能。
        \item 如果仪表指示值约为 \textbf{$0^\circ\text{C}$},则说明仪表\textbf{无}冷端补偿功能(或功能未开启)。
    \end{itemize} 
\end{solution}

\begin{problem}
检定配热电偶的温度显示仪表时,常用标准电势输入替代实际热电偶的热电
势。现要检定K 型温度显示仪表在700℃时的准确值,问需要输入多大的标
准电势(环境温度为25℃)。
\end{problem}
\begin{solution}

\textbf{已知条件:}
    \begin{itemize}
        \item 检定点温度 $T = 700^\circ\text{C}$
        \item 环境温度 $T_0 = 25^\circ\text{C}$
        \item 查 K 型热电偶分度表:
        $$ E_K(700^\circ\text{C}, 0) \approx 29.129\,\text{mV} $$
        $$ E_K(25^\circ\text{C}, 0) \approx 1.000\,\text{mV} $$
    \end{itemize}

    \textbf{计算过程:}
    \begin{align*}
        V_{\text{std}} &= E_K(T, 0) - E_K(T_0, 0) \\
        &= 29.129\,\text{mV} - 1.000\,\text{mV} \\
        &= 28.129\,\text{mV}
    \end{align*}

    \textbf{答案:} 需要输入的标准电势约为 \textbf{$28.129\,\text{mV}$}。 
\end{solution}
\clearpage
\section{压力、液位、流量实验}
\subsection{数字式压力校验仪实验}
\subsubsection{实验目的}
\begin{enumerate}
  \item 了解弹簧管压力表的工作原理、构造、装配和调整方法。
  \item 掌握各类压力仪表与压力变送器的校验方法。
  \item 了解弹簧管压力表的精度等级划分,机械仪表变差的测定。
\end{enumerate}
\subsubsection{实验器材}
\begin{enumerate}
  \item 弹簧管压力表 1 只
  \item 数字式压力校验仪一台
  
  图\ref{qianban}为校验仪的前面板示意。

  \begin{enumerate}
    \item 压力微调机构
    
    在校准被校仪表是可以左右旋转压力微调机构,使输出压力达最佳值。
    \item 手压泵
    
    推动手压泵可产生一定的压力值,使用手压泵时,首先了解清楚本仪器与
被校仪表的测量范围,切忌超过校验仪表与仪器的量程范围!!增压时,应注视
显示屏的压力变化。当仪器出现嘟嘟声报警时,应立即停止增压,并打开放气阀
泄压。
    \item 放气阀
    
    放气阀为校验仪提供增压、泄压而设计。增压时,应顺时针方向旋转到底,
放气阀关闭;泄压时,逆时针方向旋转开启放气阀。
    \item 压力接头
    
    供输入/输出压力用。
    \item 提手
    
    提手供校验仪外出时使用。
  \end{enumerate}

  \begin{figure}[!htbp]
    \centering
    \includegraphics[width=0.3\textwidth]{qianban.png}
    \caption{前面板}
    \label{qianban}
  \end{figure}
\end{enumerate}

\subsubsection{实验步骤}
\textbf{压力表检定一般至少检定十点,这些点应均匀地分布在整个量程范围内。}

了解数字式压力校验仪的特性和优点、技术指标、可校验参数(如压力、电
流、电压),掌握零点调整、基本参数(比如量程、单位)设定、校表步骤,最
后用其校定普通压力表。

上行程时,用打气手柄进行充气,当液晶屏显示值接近设定点,可以通过微
调旋钮调节至工作点。降程时,缓慢的打开放气阀,当接近设定点时,利用微调
旋钮外旋减压。
\subsubsection{数据记录与处理}
\begin{enumerate}
  \item \textbf{数据记录}
  \begin{table}[htbp]
\centering
\caption{实验数据记录表}
% 定义列格式:共8列,全部居中,两边及中间都有竖线
\begin{tabular}{|c|c|c|c|c|c|c|c|}
\hline
% 第一行表头
\multirow{3}{*}{编号} & \multicolumn{2}{c|}{被校压力表读数} & \multicolumn{2}{c|}{标准压力表读数} & \multicolumn{2}{c|}{\multirow{2}{*}{误差/MPa}} & \multirow{3}{*}{变差} \\
& \multicolumn{2}{c|}{/MPa} & \multicolumn{2}{c|}{/MPa} & \multicolumn{2}{c|}{} & \\
\cline{2-7} % 绘制中间部分的横线
% 第二行表头(升程/降程)
& 升程 & 降程 & 升程 & 降程 & 升程 & 降程 & \\
\hline
1 & 0.06 & 0.06 & 0.0626 & 0.0515 & -0.0026 & 0.0085 & 0.0111 \\
\hline
2 & 0.12 & 0.12 & 0.1214 & 0.1123 & -0.0014 & 0.0077 & 0.0091 \\
\hline
3 & 0.18 & 0.18 & 0.1796 & 0.1691 & 0.0004 & 0.0109 & 0.0105 \\
\hline
4 & 0.24 & 0.24 & 0.2393 & 0.2311 & 0.0007 & 0.0089 & 0.0082 \\
\hline
5 & 0.30 & 0.30 & 0.2972 & 0.2885 & 0.0028 & 0.0115 & 0.0087 \\
\hline
6 & 0.36 & 0.36 & 0.3572 & 0.3493 & 0.0028 & 0.0107 & 0.0079 \\
\hline
7 & 0.42 & 0.42 & 0.4162 & 0.4065 & 0.0038 & 0.0135 & 0.0097 \\
\hline
8 & 0.48 & 0.48 & 0.4736 & 0.4670 & 0.0064 & 0.0130 & 0.0066 \\
\hline
9 & 0.54 & 0.54 & 0.5341 & 0.5278 & 0.0059 & 0.0122 & 0.0063 \\
\hline
10 & 0.60 & 0.60 & 0.5907 & 0.5870 & 0.0093 & 0.0130 & 0.0037 \\
\hline
\end{tabular}
\end{table}
 \item \textbf{精确度等级确定}
 
 比较所有数据,可以得到最大绝对误差 $\Delta_{max} = 0.0135 \text{ MPa}$。据此计算最大引用误差:
 $$\gamma = \frac{0.0135}{0.60} \times 100\% = 2.25\%$$
 根据国家标准,该仪表的精确度等级为2.5级。

 \item \textbf{校正曲线}
     \begin{figure}[!htbp]
        \centering
        \includegraphics[width=0.8\textwidth]{压力校正曲线.png}
        \caption{压力校正曲线}
    \end{figure}
\end{enumerate}


\subsubsection{思考题}
\begin{problem}
弹簧管压力表为什么要在测量上限处进行耐压检定?
\end{problem}
\begin{solution}
主要有两方面原因:
\begin{enumerate}
    \item \textbf{消除弹性后效}:通过在测量上限受压,可以让弹簧管的材料微观结构产生必要的应力松弛,消除弹性滞后和后效现象,使其弹性特性趋于稳定,从而保证测量的重复性。
    \item \textbf{检查耐压强度与密封性}:验证弹簧管及其焊接部位在承受最大工作压力时,是否会出现泄漏、破裂或发生永久性的塑性变形。
\end{enumerate}
\end{solution}

\begin{problem}
检定装置在校验压力表时会带来哪些附加误差?
\end{problem}
\begin{solution}
主要包含以下几类附加误差:
\begin{enumerate}
    \item \textbf{标准器误差}:标准表或标准源自身精度等级带来的系统误差。
    \item \textbf{液柱高度差误差}:当被检表与标准表安装高度不同时,传压介质(特别是油介质)的重力会产生附加压力差。
    \item \textbf{环境误差}:环境温度偏离标准温度($20\pm 5^\circ\text{C}$)时,会导致弹性模量变化,引起温漂。
    \item \textbf{读数误差}:观察者视线未与表盘垂直产生的视差,以及指针估读时的随机误差。
\end{enumerate}
\end{solution}

\clearpage

\subsection{液位变送器仪表精度校验实验}
\subsubsection{实验目的}
\begin{enumerate}
  \item 通过实验了解液位测量的基本方法、工作原理及使用与校验方法;
  \item 仪表误差分析方法;
  \item 了解差压变送器的工作原理及使用方法;
  \item 了解零点迁移、满度调校等基本概念;
\end{enumerate}
\subsubsection{实验器材}
CS1100 过程控制实验装置。
\subsubsection{实验原理}
CS1100 过程控制实验装置对象系统包含有:两个液位计量罐、一个温度搅
拌罐,一个流量计量罐,一个储水箱。系统动力支路:由循环水泵、电动调节阀、
电动球阀等组成;检测变送和执行元件有:差压变送器、压力表、电磁流量计、
电动调节阀等。

CS1100 装置的控制系统配备西门子触摸屏作为人机操作界面。通过触摸屏
控制面板上的操作按键直接控制水泵、开关阀等设备,调节SV、MV、PID 等控
制参数,也可以通过上位机组态软件进行监控,观察被控参数的实时曲线、历史
曲线,SV 设定值、PV 测量值、OP 输出值、各实验都设有动态流程图、及被测
参数动态显示及变化棒图显示系统流程图,根据实际需要实现编辑、下载、打印
等功能。

本次实验使用差压变送器来检测液位高度,并与实际液位标尺值进行比较,
计算出差压变送器的测量精度等性能指标。

差压变送器的工作原理:当被测介质(液体)的压力作用于传感器时,压力
传感器将压力信号转换成电信号,经归一化差分放大和输V/A 电压、电流转换
器,转换成与被测介质(液体)的液位压力成线性对应关系的4~20mA 标准电流
输出信号。接线如图\ref{chayabiansongqi}所示。
  \begin{figure}[!htbp]
    \centering
    \includegraphics[width=0.3\textwidth]{差压变送器接线图.png}
    \caption{差压变送器接线图}
    \label{chayabiansongqi}
  \end{figure}

液位变送器实验的流程图如下:
  \begin{figure}[!htbp]
    \centering
    \includegraphics[width=0.8\textwidth]{液位变送器试验流程图.png}
    \caption{液位变送器示意图}
  \end{figure}

\subsubsection{实验步骤}
\begin{enumerate}
  \item 启动1\#输送泵,打开SV102,,确保进水手阀畅通,将2\#计量罐加满水。
  
  \begin{figure}[!htbp]
        \centering
        \begin{minipage}[b]{0.45\linewidth}
            \centering
            \includegraphics[width=0.9\textwidth]{输送泵和 SV102 操作界面1.png}
             
        \end{minipage}%
        \begin{minipage}[b]{0.45\linewidth}
            \centering
            \includegraphics[width=0.9\textwidth]{输送泵和 SV102 操作界面2.png}    
        \end{minipage}
        \caption{1\#输送泵和SV102 操作界面}
    \end{figure}

  \item 关停1\#输送泵,点击最下行菜单栏“2\#计量罐液位”,进入实验画面,记录当前测量值和标尺值。
  
      \begin{figure}[!htbp]
        \centering
        \includegraphics[width=0.8\textwidth]{计量罐液位实验界面.jpeg}
        \caption{2\#计量罐液位实验界面}
    \end{figure}

  \item 手动调节2\#计量罐底部排水阀门开度,缓慢降低液柱,在全量程范围内均匀选取10 个校验点,记录标尺值和液位测量值,并在表中记录实验数据。
    \begin{figure}[!htbp]
        \centering
        \includegraphics[width=0.7\textwidth]{计量罐底部排水阀.png}
        \caption{计量罐底部排水阀}
    \end{figure}
\end{enumerate}
\subsubsection{数据记录与处理}
\begin{enumerate}
  \item \textbf{数据记录}
  液位显示零点读数为0.5
\begin{table}[htbp]
\centering
\caption{液位测量数据记录表}
\resizebox{\textwidth}{!}{ 
    \begin{tabular}{|c|c|c|c|c|c|c|c|c|c|c|}
    \hline
     & 1 & 2 & 3 & 4 & 5 & 6 & 7 & 8 & 9 & 10 \\
    \hline
    标尺读数 & 22.00 & 20.00 & 18.00 & 16.00 & 14.00 & 12.00 & 10.00 & 8.00 & 6.00 & 4.00 \\
    \hline
    液位显示值 & 21.9 & 19.9 & 18.1 & 16.1 & 14.2 & 12.1 & 10.4 & 8.3 & 6.2 & 4.5 \\
    \hline
    校准后液位显示值 & 21.4 & 19.4 & 17.6 & 15.6 & 13.7 & 11.6 & 9.9 & 7.8 & 5.7 & 4 \\
    \hline
    相对误差 & 2.80\% & 3.09\% & 2.27\% & 2.56\% & 2.19\% & 3.45\% & 1.01\% & 2.56\% & 5.26\% & 0.00\% \\
    \hline
    \end{tabular}
}
\end{table}
  \item \textbf{参数曲线}
      \begin{figure}[!htbp]
        \centering
        \includegraphics[width=0.8\textwidth]{标尺读数与数显仪读数.png}
        \caption{标尺读数与数显仪读数曲线}
    \end{figure}
  \item \textbf{误差分析}
  
  求取原始数据的曲线可得斜率$k=0.966$,这说明传感器不仅存在零点漂移的问题,还存在灵敏度偏小的问题。

  因此如果后续要进行进一步校准,应当对于斜率也进行校准。
  
\end{enumerate}
\subsubsection{思考题}
\begin{problem}
差压变送器的引压点对测量的结果有无影响?什么是正迁移和负迁移?
如何确定引压点的位置?
\end{problem}
\begin{solution}
\begin{enumerate}
    \item \textbf{引压点的影响}:有影响。引压点与变送器安装位置的高度差会形成液柱,产生附加的静压力($\rho g h$),导致测量结果偏大或偏小,因此必须通过“零点迁移”来消除。
    \item \textbf{正迁移与负迁移}:
    \begin{itemize}
        \item \textbf{正迁移}:当变送器安装位置低于取压点(敞口容器测量),正压侧受液柱压力使得测量值偏大,需要将测量范围向正方向移动(扣除固定静压),称为正迁移。
        \item \textbf{负迁移}:当负压侧充满了隔离液(如密闭容器带隔离液测量),负压侧压力大于正压侧液柱压力,使得差压为负值,需要将测量范围向负方向移动,称为负迁移。
    \end{itemize}
    \item \textbf{引压点位置确定原则}:
    \begin{itemize}
        \item \textbf{测气体}:引压点在管道上半部(防止积液)。
        \item \textbf{测液体}:引压点在管道下半部与水平中心线成 $0^\circ \sim 45^\circ$ 夹角范围内(防止积气,且避开底部沉淀)。
        \item \textbf{测蒸汽}:引压点在管道上半部或侧面,且需加装冷凝罐。
    \end{itemize}
\end{enumerate}
\end{solution}

\begin{problem}
与敞开容器相比密闭容器在测液位时需注意什么问题?
\end{problem}
\begin{solution}
密闭容器测量液位时,液体表面上方存在气相静压力,不能直接通大气,需注意以下问题:
\begin{enumerate}
    \item \textbf{气相补偿}:差压变送器的负压室(低压侧)必须通过引压管连接到容器的气相空间,以抵消气相压力对测量的影响($\Delta P = P_{\text{底部}} - P_{\text{顶部}} = \rho g h$)。
    \item \textbf{冷凝液的影响(干腿 vs 湿腿)}:
    \begin{itemize}
        \item 如果气相介质不冷凝(或无腐蚀),可使用空管连接(干腿)。
        \item 如果气相介质易冷凝(如蒸汽),负压侧引压管内会积聚液体,此时必须预先灌满隔离液,并因此导致负压侧压力常大于正压侧,必须进行\textbf{100\% 负迁移}。
    \end{itemize}
\end{enumerate}
\end{solution}

\clearpage
\subsection{电磁流量计校验实验}
\subsubsection{实验目的}
\begin{enumerate}
  \item 了解电磁流量计的工作原理、外形构造及使用方法;
  \item 掌握流量仪表检定的基本方法;
  \item 掌握流量检测仪表精度等级的校验方法。
\end{enumerate}
\subsubsection{实验器材}
CS1100 过程控制实验装置。
\subsubsection{实验原理}
电磁流量计是根据法拉第电磁感应定律进行流量测量的流量计。当导体在磁
场中作切割磁力线运动时,在导体中会产生感应电势,感应电势的大小与导体在
磁场中的有效长度及导体在磁场中作垂直于磁场方向运动的速度成正比。同理,
导电流体在磁场中作垂直方向流动而切割磁感应力线时,也会在管道两边的电极
上产生感应电势。感应电势的方向由右手定则判定,感应电势的大小由下式确定:
$$Ex=BDv$$
式中$Ex$为感应电动势;$B$为磁感应强度;$D$为管道内径;$v$为液体的平均流速。
    \begin{figure}[!htbp]
        \centering
        \includegraphics[width=0.5\textwidth]{电磁流量计原理示意图.png}
        \caption{电磁流量计原理示意图}
    \end{figure}

    然而体积流量$Q_v$等于流体的流速$v$与与管道截面积$\pi D^2/4$的乘积,可得:
  $$Q_v=(\pi D /4B) \times Ex$$

  由上式可知,在管道直径$D$己定且保持磁感应强度$B$不变时,被测体积流量
与感应电势呈线性关系。若在管道两侧各插入一根电极,就可引入感应电势$Ex$测量此电势的大小,就可求得体积流量。

据法拉第电磁感应原理,在与测量管轴线和磁力线相垂直的管壁上安装了一
对检测电极,当导电液体沿测量管轴线运动时,导电液体切割磁力线产生感应电
势,此感应电势由两个检测电极检出,数值大小与流速成正比例,其值为:
$$E=B \cdot v \cdot D \cdot K$$

传感器将感应电势$E$作为流量信号,传送到转换器,经放大,变换滤波等信号
处理后,成为标准的变送信号输出。
流量计校验实验的流程图如下:
    \begin{figure}[!htbp]
        \centering
        \includegraphics[width=0.5\textwidth]{液位变送器实验流程图2.png}
        \caption{液位变送器实验流程图}
    \end{figure}
\subsubsection{实验步骤}
\begin{enumerate}
  \item 启动1\#输送泵,打开SV102,,确保进水手阀畅通,向2\#计量罐加水。
    \begin{figure}[!htbp]
        \centering
        \begin{minipage}[b]{0.3\linewidth}
            \centering
            \includegraphics[width=0.9\textwidth]{输送泵和 SV102 操作界面3.png}
             
        \end{minipage}%
        \begin{minipage}[b]{0.3\linewidth}
            \centering
            \includegraphics[width=0.9\textwidth]{输送泵和 SV102 操作界面4.png}
             
        \end{minipage}
        \caption{1\#输送泵和SV102 操作界面}
    \end{figure}
  \item 确认“2\#计量罐出口流量控制”为手动状态,在控制面板中将MV 设置为100\%,使流入温度搅拌罐的流量最大。等待搅拌罐灌满水,自动停止注水。
      \begin{figure}[!htbp]
        \centering
        \includegraphics[width=0.4\textwidth]{计量罐出口流量控制面板.png}
        \caption{2\#计量罐出口流量控制面板}
    \end{figure}
  \item 全开3\#计量罐底部排水阀,启动2\#输送泵。确认“搅拌釜液位控制”为手动状态,在控制面板中将MV 设置为20\%,流量稳定后读取电磁流量计的读数。
  
      \begin{figure}[!htbp]
        \centering
        \begin{minipage}[b]{0.3\linewidth}
            \centering
            \includegraphics[width=0.9\textwidth]{输送泵和搅拌釜液位控制面板1.png}
             
        \end{minipage}%
        \begin{minipage}[b]{0.3\linewidth}
            \centering
            \includegraphics[width=0.9\textwidth]{输送泵和搅拌釜液位控制面板2.png}
             
        \end{minipage}
        \caption{2\#输送泵和搅拌釜液位控制面板}
    \end{figure}
  \item 迅速关断3\#计量罐底部排水阀同时开始计时,并记录当时液位高度值。20s 后停止计时,同时记录当时液位高度值。完成后尽快停止2\#输送泵,以防止搅拌罐液位过低。
  \item 将“搅拌釜液位控制”MV 值设置为40\%,60\%,重复步骤3、4。(受搅拌罐容量限制,MV 为40\%时计时25s,MV 为60\%时计时20s)

\end{enumerate}
\subsubsection{数据记录与处理}
计算流量通过液面高度的变化求得。

\begin{table}[htbp]
\centering
\caption{流量测量实验数据记录表}
% 定义列格式:左对齐(描述) + 3个居中(数据),带竖线
\begin{tabular}{|l|c|c|c|}
\hline
 & 1 & 2 & 3 \\
\hline
搅拌釜液位控制 MV 值 & 20 & 40 & 60 \\
\hline
计量罐液位起始高度 (cm) & 0.5 & 0.8 & 2 \\
\hline
计量罐液位最终高度 (cm) & 8.1 & 10 & 10.8 \\
\hline
时间长度 (s) & 30 & 25 & 20 \\
\hline
计算流量 (L/min) & 3.192 & 4.6368 & 5.544 \\
\hline
标准流量 (L/min) & 3 & 4.3 & 5.3 \\
\hline
电磁流量计 (L/min) & 3.04 & 4.28 & 5.28 \\
\hline
误差 & 0.04 & 0.02 & 0.02 \\
\hline
\end{tabular}
\end{table}

3\#计量罐的内长21cm,内宽10cm,溢流管外径2cm。
\subsubsection{思考题}
这是为您整理的关于电磁流量计及实验误差分析的标准解答,可以直接填入 LaTeX 的 solution 环境中。

1. LaTeX 代码
Code snippet

\begin{problem}
查找资料,了解电磁流量计的测量范围、使用的介质、精度等级、对测
量条件的要求等。
\end{problem}
\begin{solution}
基于法拉第电磁感应定律,电磁流量计的特性如下:
\begin{enumerate}
    \item \textbf{测量范围}:量程比极宽(通常可达 10:1 至 100:1),适应流速范围通常在 $0.1\,\text{m/s} \sim 10\,\text{m/s}$ 之间。
    \item \textbf{使用介质}:必须是\textbf{导电液体}(电导率需 $> 5\,\mu\text{S/cm}$),如水、酸碱溶液、泥浆等。不能测量气体、蒸汽、油品或纯净水。
    \item \textbf{精度等级}:较高,通常为 0.5 级,高精度产品可达 0.2 级。
    \item \textbf{测量条件要求}:
    \begin{itemize}
        \item \textbf{满管要求}:测量管道必须完全充满液体。
        \item \textbf{直管段}:为保证流场稳定,通常要求上游直管段长度 $\ge 5D$,下游 $\ge 3D$(D为管径)。
        \item \textbf{接地}:流量计外壳及介质必须良好接地,以消除干扰电压。
    \end{itemize}
\end{enumerate}
\end{solution}

\begin{problem}
通过实验分析该装置检定流量仪表时误差来源有那些?如何避免与消
除?
\end{problem}
\begin{solution}
结合容积法(计量罐)检定电磁流量计的实验过程,分析如下:
\begin{enumerate}
    \item \textbf{误差来源}:
    \begin{itemize}
        \item \textbf{标准器误差}:计量罐的刻度尺分辨率有限(读数误差),以及秒表计时的反应滞后(时间测量误差)。
        \item \textbf{启停同步误差}:液位开始变化的瞬间与秒表计时的启动瞬间无法完全同步(动态误差)。
        \item \textbf{流场干扰}:泵的脉动导致流速不稳定,或管道内存在气泡(电磁流量计最忌讳气泡)。
    \end{itemize}
    \item \textbf{避免与消除方法}:
    \begin{itemize}
        \item \textbf{延长测量时间}:增加计量时间(如 $>60$s),降低启停计时误差对总结果的权重。
        \item \textbf{保证直管段}:确保流量计前后有足够的直管段,安装整流器以稳定流场。
        \item \textbf{多次测量求平均}:在同一流量点进行多次测量取平均值,消除随机误差。
        \item \textbf{静态读数法}:先切换阀门将流体引入计量罐,待液面完全平静后再读取高度,避免液面波动带来的读数误差。
    \end{itemize}
\end{enumerate}
\end{solution}
\clearpage
\section{智能产线传感与检测实验}

\end{document}
